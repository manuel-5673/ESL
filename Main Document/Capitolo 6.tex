\documentclass[11pt,openany]{book}
\usepackage[utf8]{inputenc}
\usepackage[spanish]{babel}
\usepackage{amsmath,times}
\usepackage{listings}
\usepackage{amssymb,amsmath}
\usepackage{graphicx}
\graphicspath{{../Figures/}}
\usepackage{amsthm}
\usepackage{enumerate}
\usepackage[
	backend=biber,
	style=alphabetic,
	citestyle=alphabetic
]{biblatex}
\usepackage[a4paper]{geometry}


\addbibresource{../References/referencias.bib}

\lstset{
  basicstyle=\ttfamily,
  mathescape
}
\spanishdecimal{.}


\begin{document}

\tableofcontents

\chapter{Metodi di levigatura del kernel}

In questo capitolo descriviamo una classe di tecniche di regressione che ottengono flessibilità nella stima della funzione di regressione $f(X)$ nel dominio $\mathbb{R}^p$, adattando separatamente un modello diverso ma semplice in ogni ponto di input $x_0$. Questo viene fatto usando solo quelle osservazioni vicine al punto obiettivo $x_0$ per adattare il modello semplice, e in modo tale da la funzione stimata risultante $\hat{f}(X)$ è liscia in $\mathbb{R}^p$. 

\subsection{}


%\printbibliography

\end{document}
