\documentclass[11pt,openany]{book}
\usepackage[utf8]{inputenc}
\usepackage[italian]{babel}
\usepackage{amsmath,times}
\usepackage{listings}
\usepackage{amssymb,amsmath}
\usepackage{graphicx}
\graphicspath{{../Figures/}}
\usepackage{amsthm}
\usepackage{enumerate}
\usepackage[
	backend=biber,
	style=alphabetic,
	citestyle=alphabetic
]{biblatex}
\usepackage[a4paper]{geometry}

\newenvironment{nalign}{
    \begin{equation}
    \begin{aligned}
}{
    \end{aligned}
    \end{equation}
    \ignorespacesafterend
}


\addbibresource{../References/referencias.bib}

\lstset{
  basicstyle=\ttfamily,
  mathescape
}


\begin{document}

\tableofcontents

\chapter{Metodi lineari di classificazione}

\section{Introduzione}

In questo capitolo rivisitiamo il problema di classificazione e ci concentriamo in metodi lineari di classificazione. Dato che il nostro predittore $G(x)$ prende valori di un insieme discreto $\mathcal{G}$, possiamo sempre dividere lo spazio di input in una collezione di regioni secondo la classificazione. Abbiamo visto nel Capitolo 2 che le frontiere di questi regioni possono essere ruvide o lisce, dipendendo della funzione di predizione. Per una classe importante di procedure, queste \textit{frontiere di decisione} sono lineari; questo è ciò che intendiamo con metodi lineari di classificazione.

Ci sono diversi modi in cui le frontiere di decisione lineari possono essere trovate. Nel Capitolo 2 abbiamo adattato modelli di regressione lineare alle variabili indicatori di classi, e dopo classifichiamo secondo il \textbf{largest fit}. Supponiamo che ci sono $K$ classi, etichettate come $1, 2, \dots, K$ per comodità, ed il modello lineare adattato per la $k$-esima variabile indicatore di risposta sia $\hat{f}_k(x) = \hat{\beta}_{k0} + \hat{\beta}_k^T x$. La frontiera di decisione tra la classe $k$ e $l$ è il insieme di punti per cui $\hat{f}_k (x) = \hat{f}_l (x)$, cioè, il insieme $\{ x : \left( \hat{\beta}_{k0} - \hat{\beta}_{l0} \right) + \left( \hat{\beta}_{k} - \hat{\beta}_{l} \right)^T x = 0 \}$, che è un insieme affine o un iperpiano\footnote{In senso stretto, un iperpiano attraversa l'origine, mentre un insieme affine non necessariamente. A volte ignoriamo la distinzione e intendiamo iperpiani in generale}. Siccome lo stesso è vero per ogni paio di classi, lo spazio di input è diviso in regioni di classificazione costante, con frontiere di decisione che sono iperplanare a tratti. Questo approccio con regressione fa parte di una classe di metodi che modellano \textit{funzioni discriminanti} $\delta_k (x)$ per ogni classe, e dopo classificano $x$ alla classe con il valore più grande nella sua funzione discriminante. Metodi che modellano le probabilità posteriori $\text{Pr}(G = k | X = x)$ appartengono anche a questa classe. Chiaramente, se $\delta_k (x)$ o $\text{Pr}(G = k | X = x)$ sono lineare in $x$, allora le frontiere di decisione saranno anche lineare.

In realtà, tutto ciò di cui abbiamo bisogno è che alcuna trasformazione monotona di $\delta_k$ o $\text{Pr}(G = k | X = x)$ sia lineare affinché le frontiere di decisione siano lineari. Per esempio, se ci sono due classi, un modello popolare per le probabilità posteriori è
\begin{nalign} \label{eq4-1}
\text{Pr}(G = 1 | X = 1) & = \frac{e^{\beta_0 + \beta^T x}}{1 + e^{\beta_0 + \beta^T x}},\\
\text{Pr}(G = 2 | X = 1) & = \frac{1}{1 + e^{\beta_0 + \beta^T x}}.
\end{nalign}
La trasformazione monotona qui è la trasformazione \textit{logit}: $\ln[p/(1 - p)]$\footnote{Se $p_1 = \text{Pr}(G = 1 | X = 1)$ e $p_2 = \text{Pr}(G = 2 | X = 1)$, poi abbiamo che $\ln[p_1/(1 - p_1)] = \beta_0 + \beta^T x$ e $\ln[p_2/(1 - p_2)] = - \left( \beta_0 + \beta^T x \right)$.}, e infatti vediamo che
\begin{equation} \label{eq4-2}
\ln \frac{\text{Pr}(G = 1 | X = 1)}{\text{Pr}(G = 2 | X = 1)} = \beta_0 + \beta^T x.
\end{equation}
La frontiera di decisione è il insieme di punti dove le log-probabilità sono zero, e questo è un iperpiano definito da $\{ x | \beta_0 + \beta^T x = 0 \}$. Discutiamo qui due metodi molto popolari ma diversi che risultano in log-probabilità lineari o \textit{logits}: il analisi di discriminante lineare e la regressione logistica lineare. Sebbene sono diversi nella sua derivazione, la differenza essenziale tra loro è nel modo in cui la funzione lineare si adatta ai dati di addestramento.

Un approccio più diretto è modellare esplicitamente le frontiere tra le classi come lineare. 

%\printbibliography

\end{document}
